\documentclass[12pt]{article}
\usepackage{amsmath,amssymb}
\textheight 240mm
\textwidth  170mm
\oddsidemargin  0mm
\evensidemargin 0mm
\topmargin -20mm

\usepackage[utf8]{inputenc}
\usepackage{graphicx}
\usepackage{float}
\usepackage{titling}
\usepackage{geometry}
\usepackage{color}

\geometry{ a4paper,
	left=30mm,
	right=30mm,
	top=20mm,
}

\addtolength{\headheight}{0.2pt}
%\setlength{\droptitle}{-8em} 

\title{Music Genre Classification}
\author{Abhijit Suresh, Paria Rezaeinia, Sahana Sadogopan}

\begin{document}
	\maketitle
\begin{abstract}
	Here goes the abstract: : \textcolor{red}{Paria}
\end{abstract}
	
\section{Introduction}
%_________________________________________________________________
You will properly define the genre classification problem, and
indicate a few references to the literature.
%_________________________________________________________________
explaining the problem, the current and common methods to solve this problem. the way that we approach it, the algorithms that we use and the reason we use these algorithms. a very brief overview of the results. The organization of the paper. : \textcolor{red}{Paria}
\section{Dimensionality Reduction}
%_________________________________________________________________
Describe your dimension reduction technique, and justify why it 
is appropriate to use it in this context. You should explain what 
performance is expected. 
%_________________________________________________________________
I suggest we give some background to dimensionality reduction and also mention the Johnson-Lindestrauss theorem here. : \textcolor{red}{Sahana}
\subsection{mfcc} \textcolor{red}{Sahana}

\subsection{PCA} \textcolor{red}{Sahana}

\subsection{k-means} \textcolor{red}{Abhijit}

\subsection{Multidimensional Scaling} \textcolor{red}{Abhijit}

\subsection{Modified Gaussian Mixture} \textcolor{red}{Paria}

\section{Distance Metrics}
Explain what are the available distance in the space of songs.
Describe your distance and any pre-processing performed before
computing the distance. : \textcolor{red}{Abhijit}
%_________________________________________________________________
\subsection{Minowski distance}\textcolor{red}{Abhijit}
\subsection{Earth Movers distance}\textcolor{red}{Abhijit}
\subsection{Euclidean distance}\textcolor{red}{Paria}
\subsection{Kullback-Leibler distance (KL) distance}\textcolor{red}{Paria}

\section{Statiscal learning}
%_________________________________________________________________
Explain how the training data help find the genre of an unknown
song. This could be as simple as finding the closest song among all
the songs for which you know the genre. Or it could involve more
sophisticated methods. \textcolor{red}{Paria}
%_________________________________________________________________

\subsection{kNN}\textcolor{red}{Paria}

\subsection{Modified-kNN}\textcolor{red}{Paria}

\subsection{Neural Network}\textcolor{red}{Sahana}
\section{Experiments}
%_________________________________________________________________
Describe the experiments, and include the confusion matrix. Discuss
the influence of the various parameters, and describe how the optimal
parameters were chosen. Include the computation time for your method. : \textcolor{red}{Sahana}
%_________________________________________________________________
\section{Discussion}
%_________________________________________________________________
Provide a critique of the approach and discuss any potential
improvement. Discuss the ability of your approach to classify
non-classical into the five remaining genres. \textcolor{red}{Abhijit}
\begin{thebibliography}{9}
	\bibitem{bingham} 
	Bingham, E., Mannila, H. (2001, August). Random projection in dimensionality reduction: applications to image and text data. In Proceedings of the seventh ACM SIGKDD international conference on Knowledge discovery and data mining (pp. 245-250). ACM.
	\bibitem{pamalk} Pampalk, E. (2006). Computational models of music similarity and their application in music information retrieval. na.
	\bibitem{eronen} Eronen, A. (2003, July). Musical instrument recognition using ICA-based transform of features and discriminatively trained HMMs. In Signal Processing and Its Applications, 2003. Proceedings. Seventh International Symposium on (Vol. 2, pp. 133-136). IEEE.
	\bibitem{logan} Logan, B., Salomon, A. (2001). A content-based music similarity function. Cambridge Research Labs-Tech Report.
	\bibitem{george} George, T., Georg, E., Perry, C. (2001, October). Automatic musical genre classification of audio signals. In Proceedings of the 2nd International Symposium on Music Information Retrieval, Indiana.
\end{thebibliography}
\end{document}